\documentclass[]{article}
\usepackage{amsmath}		% For generic math symbols
\usepackage{amssymb}		% For mathbb
\usepackage{enumerate}		% For lists indexed by letters
\usepackage{bm}				% For bold letters
\usepackage{enumitem}		% So we can resume counting problem numbers after
% interrupting with text
\usepackage{hyperref}		% For clickable URL links
\usepackage{url}			% So file names won't create hboxes

\setlength{\parindent}{0pt}	% Turns off indentation


% Set some useful commands
\newcommand{\half}{\frac{1}{2}}			% 1/2
\newcommand{\R}{\mathbb{R}}				% Reals symbol
\newcommand{\bbm}{\begin{bmatrix}}		% Begin bmatrix environment
\newcommand{\ebm}{\end{bmatrix}}		% End bmatrix environment
\newcommand{\x}{\bm{x}}					% Bold (vector) x
\newcommand{\y}{\bm{y}}					% Bold (vector) y
\newcommand{\A}{\bm{A}}					% Bold (matrix) A
\newcommand{\I}{\bm{I}}					% Bold (matrix) I
\newcommand{\vspan}{\mathrm{span}}		% To use the word span in math mode

% Place this command after each problem, before solution (examples below)
\newcommand{\solution}{\vskip 0.5cm \textbf{\large Solution:} \\}


\title{AMATH 352: Problem Set 5}
\author{Your name here}

\begin{document}

\maketitle
    {\Large \textbf{Due: Friday February 17, 2017}} \\


    \section*{Matrices:}
    \begin{enumerate}
	\item Consider a real matrix $\A\in\R^{m\times n}$ with $m\geq n$ which is full rank, i.e. rank($\A$)$=n$.
	  \begin{enumerate}
	  \item Find the nullspace of $\A$.
	  \item Show that for any vector $\x\in\R^n,~\x^T\A^T\A\x=\|\A\x\|^2_2$.
      \item Show that the matrix $\A^T\A$ is invertible. Hint: show the nullspace of $\A^T\A$ is $\{\bm{0}\}$ by assuming $\bm{z}\in\mathrm{null}(\A^T\A)$ then using the previous parts of this question to show that $\bm{z}=\bm{0}$. (Second hint: try multiplying by $\bm{z}^T$. You will also need one of the defining properties of a norm).
	  \end{enumerate}

	  \solution
	  \begin{enumerate}
	  \item By rank-nullity, $rank(A) + dim(null(\A)) = n$, so the
        dimension of the nullspace is 0 which implies $null(\A) =
        \{\bm{0}\}$.
	  \item TODO Your solution goes here.
	  \item TODO Your solution goes here.
	  \end{enumerate}
	  
	\item Suppose $\A\in\R^{n\times n}$ is invertible. Find the inverse of $\A^T\A$ and show that it is symmetric.

	  \solution
	  Using the associative property of matrix multiplication, we can see that the inverse of $\A^T\A$ is $\A^{-1}(\A^T)^{-1}$ since
      \[
      \A^T \A \A^{-1}(\A^T)^{-1} = \A^T \I (\A^T)^{-1} = \I
      \]
      But the properties of the matrix inverse allows us to express
      $(\A^T)^{-1}$ as $(\A^{-1})^T$, so the inverse is equivalently
      $\A^{-1}(\A^{-1})^T$, in otherwords, it's the product of a matrix
      with its transpose. Any such matrix product must be symmetric,
      since for $\bm{C} = \bm{B}\bm{B}^T$,
      \[
      \bm{C}_{ij} = \sum_{k=1}^{m} \bm{B}_{ik} \bm{B}^T_{kj}
      \]
      but $\bm{B}^T_{kj} = \bm{B}_{jk}$ so
      \[
      \bm{C}_{ij} = \sum_{k=1}^{m} \bm{B}_{ik} \bm{B}_{jk} = \sum_{1}^{m} \bm{B}_{jk} \bm{B}_{ik} = \bm{C}_{ji}
      \]


	\item Suppose $\A,\bm{B}\in\R^{n\times n}$. Show that $\bm{AB}$ is invertible if and only if both $\A$ and $\bm{B}$ are invertible. This means you must show that if $\bm{AB}$ is invertible, so are $\A$ and $\bm{B}$ and you must also show that if $\A$ and $\bm{B}$ are invertible, so is $\bm{AB}$. Hint: use the determinant.

	  \solution

      The statement that $\A\bm{B}$ is invertible is equivalent to
      stating that it's determinant is non-zero. By the properties of
      the determinant, we know that $$det(\A \bm{B}) = det(\A)
      \det(\bm{B})$$ so if $det(\A\bm{B}) \neq 0$, then both $\A$ and
      $\bm{B}$ must have non-zero determinants, and are invertible. If
      either of $\A$ or $\bm{B}$ are non-invertible, then it follows
      that $det(\A\bm{B}) = 0$, and $\A\bm{B}$ is non-invertible.

	\item Suppose $\A,\bm{B},\bm{C}\in\R^{n\times n}$ are all invertible. Show that $\bm{ABC}$ is invertible by finding a matrix $\bm{D}$ such that $(\bm{ABC})\bm{D}=\bm{D}(\bm{ABC})=\bm{I}$.

	  \solution

      $\bm{D} = \bm{C}^{-1} \bm{B}^{-1} \A^{-1}$ satisfies the
      requirement, since the associative property of matrix
      multiplication shows that
      $$ \bm{ABC}\bm{C}^{-1} \bm{B}^{-1} \A^{-1} = \bm{ABI} \bm{B}^{-1} \A^{-1} = \bm{AI} \A^{-1} = \I $$
      and
      $$ \bm{C}^{-1} \bm{B}^{-1} \A^{-1} \bm{ABC} = \bm{C}^{-1} \bm{B}^{-1} \bm{IBC} = \bm{C}^{-1} \bm{IC} = \I $$
      
	\item Based on your answer to the previous problem, what do you think the inverse of $\A_1\A_2\cdots \A_k$ would be, assuming $\A_1,\A_2,\dots,\A_k\in\R^{n\times n}$ are all invertible? You do not have to provide a proof, but you should briefly explain your reasoning.

	  \solution

      If we define $\A = \prod_{k=1}^{n} \A_k$, then $\A^{-1} =
      \prod_{k=n}^{1} \A_k^{-1}$, that is, we multiply the inverses in
      reverse order. This allows associative cancellation under either
      left- or right-multiplication by $\A^{-1}$.


	\item Let $\A\in\R^{n\times n}$ be invertible and let $\bm{B}\in\R^{n\times r}$ for some positive integer $r$. Show that $\bm{AX}=\bm{B}$ has a unique solution. Note that $\bm{X}\in\R^{n\times r}$ is a matrix.

	  \solution
	  TODO Your solution goes here.


	\item We saw in class that multiplication by an orthogonal matrix preserves lengths (with respect to the 2-norm). In this problem you will show that they also preserve the dot product and orthogonality, that is, the dot product between two vectors is the same as the dot product of any orthogonal matrix applied to the two vectors.

	  \solution
	  TODO Your solution goes here.

	  Let $\bm{U}\in\R^{n\times n}$ be an orthogonal matrix. Show that for any $\x,\y\in\R^n$
	  \begin{enumerate}
	  \item $(\bm{Ux})\cdot(\bm{Uy})=\x\cdot\y$,
	  \item $(\bm{Ux})\cdot(\bm{Uy}) = 0$ if and only if $\x\cdot \y=0$.

		(This means you must show that $(\bm{Ux})\cdot(\bm{Uy}) = 0$ implies $\x\cdot \y=0$ and that $\x\cdot \y=0$ implies $(\bm{Ux})\cdot(\bm{Uy}) = 0$).
	  \end{enumerate}

	  \solution
	  \begin{enumerate}
	  \item TODO Your solution goes here.
	  \item TODO Your solution goes here. 
	  \end{enumerate}
    \end{enumerate}

    \section*{Systems of equations}
    \begin{enumerate}[resume]
	\item \begin{figure}
	  \centering
	  \begin{tabular}{ccccc}
		Nutrient & Food 1 & Food 2 & Food 3 & Total nutrients required (mg) \\ \hline
		Vitamin C & 10 & 20 & 20 & 100 \\
		Calcium & 50 & 40 & 10 & 300 \\
		Magnesium & 30 & 10 & 40 & 200 \\ \hline
	  \end{tabular}
	  \caption{Milligrams (mg) of nutrients per unit of food}
	  \label{fig:diet}
	\end{figure}

	  A dietician is planning a meal that supplies certain quantities of vitamin C, calcium, and magnesium. Three foods will be included in the diet, their quantities measured in appropriate units. The nutrients supplied by these foods and the dietary requirements are given in Figure \ref{fig:diet}. Write a linear system of equations which represents the problem of choosing the appropriate amounts of each food that should be consumed to get the desired nutrients. Write the linear system of equations as a matrix-vector equation.

	  \solution
	  TODO Your solution goes here.


	\item For each of the following matrices, $\A$, and vectors, $\bm{b}$, determine the number of solutions to $\A\x=\bm{b}$.
	  \begin{enumerate}
	  \item $\A = \bbm 1&1\\1&1 \ebm,\quad \bm{b}=\bbm 1\\2\ebm$
	  \item $\A = \bbm 1&1\\1&-1\ebm,\quad \bm{b}=\bbm 1\\2\ebm$
	  \item $\A = \bbm 1 & 2 & 1 \\ 1&1&0\\1&2&0 \ebm,\quad \bm{b}=\bbm 4\\2\\3 \ebm$
	  \item $\A = \bbm 1&0&1\\0&1&1\\1&0&1 \ebm,\quad \bm{b}=\bbm 2\\2\\1 \ebm$
	  \item $\A = \bbm 1&0&1\\0&1&1\\1&0&1 \ebm,\quad \bm{b}=\bbm 5\\-3\\5 \ebm$
	  \end{enumerate}

	  \solution
	  \begin{enumerate}
	  \item TODO Your solution goes here.
	  \item TODO Your solution goes here.
	  \item TODO Your solution goes here.
	  \item TODO Your solution goes here.
	  \item TODO Your solution goes here.
	  \end{enumerate}
    \end{enumerate}

\end{document}
