\documentclass[]{article}
\usepackage{amsmath}		% For generic math symbols
\usepackage{amssymb}		% For mathbb
\usepackage{enumerate}		% For lists indexed by letters
\usepackage{bm}				% For bold letters
\usepackage{enumitem}		% So we can resume counting problem numbers after
% interrupting with section text

\setlength{\parindent}{0pt}	% Turns off indentation


% Define some useful commands
\newcommand{\half}{\frac{1}{2}}			% 1/2
\newcommand{\R}{\mathbb{R}}				% Reals symbol
\newcommand{\bbm}{\begin{bmatrix}}		% Begin bmatrix environment
\newcommand{\ebm}{\end{bmatrix}}		% End bmatrix environment
\newcommand{\x}{\bm{x}}					% Bold (vector) x
\newcommand{\y}{\bm{y}}					% Bold (vector) y
\newcommand{\vspan}{\mathrm{span}}		% To use the word span in math mode

% Place this command after each problem, before solution (examples below)
\newcommand{\solution}{\vskip 0.5cm \textbf{\large Solution:} \\}


\title{AMATH 352: Problem Set 3}
\author{Dave Moore, dmmfix@uw.edu}

\begin{document}

\maketitle
    {\Large \textbf{Due: Friday January 27, 2017}}


    \section*{Instructions:} Complete the following problems. Turn in a write up of these problems digitally (via Canvas) by 5:00pm of the due date.

    \hrulefill

    \vskip 1cm

    \section*{Linear dependence and independence:}
    \begin{enumerate}[resume]
	\item Determine whether the given vectors or functions are linearly independent. Justify your answers.
	  \begin{enumerate}
	  \item $\bbm -1\\-3\\3 \ebm, \bbm -1\\5\\-3 \ebm$
	  \item $\bbm 1\\1\\1 \ebm, \bbm 1\\2\\1 \ebm, \bbm 2\\1\\2 \ebm$
	  \item $f_1(x)=x^2+3$, $f_2(x) = 1-x$, and $f_3(x) = (x+1)^2$
	  \item $f_1(x)=1$, $f_2(x)=\sin(\pi x)$,and $f_3(x)=\cos(\pi x)$

	  \end{enumerate}

	  \solution
	  \begin{enumerate}
	  \item These vectors are linearly independent, since we can see
        that a linear combination equal to 0
        \[
        \alpha \bbm -1\\-3\\3 \ebm +  \beta \bbm -1\\5\\-3 \ebm = 0
        \]
        means that $-\alpha + -\beta = 0 \implies \alpha = \beta$ from
        the first component of the sum. But if that is true, the second
        component
        \begin{gather*}
          -3\alpha + 5\beta = 0 \\
          -3\alpha + 5\alpha = 0 \\
          2\alpha = 0
        \end{gather*}
        implies that $\alpha$ must be zero. Since we know that $\alpha =
        \beta$, only the trivial combination satisfies the original
        equation and the vectors are linearly independent.

        
	  \item
        \[
        \left(3\bbm 1\\1\\1 \ebm - \bbm 1\\2\\1 \ebm - \bbm 2\\1\\2
        \ebm\right) = \bbm 0\\0\\0 \ebm
        \]
        Is a non-trivial combination that yields $\bm{0}$, so these vectors are not linearly
        independent.

	  \item These elements of $\mathcal{P}^2$ are not linearly
        dependent. If we evaluate
        \[
        \alpha f_1 + \beta f_2 + \delta f_3
        \]
        with $\alpha = 1, \beta = -2, \delta = -1$, we find
        \[\begin{split}
        \alpha f_1 + \beta f_2 + \delta f_3 &= (x^2 + 3) - 2(1 - x) - (x + 1)^2 \\
        &= x^2 + 3 - 2 + 2x - (x^2 + 2x + 1) \\
        &= x^2 - x^2 + 2x - 2x + 3 - 2 - 1 \\
        &= 0
        \end{split}\]
        Since this is a non-trivial combination, the functions are
        not linearly independent.
        
	  \item Looking for $\alpha, \beta, \delta \in \R$ such that
        \[
        \alpha + \beta sin(\pi x) + \delta cos(\pi x) = 0
        \]
        This must hold $\forall x \in \R$. If we take $x = 0$, we
        find that
        \[
        \alpha + \beta 0 + \delta 1 = 0 \implies \delta = -\alpha
        \]
        If we take $x = 1$, we find
        \[
        \alpha + \beta 1 + \delta 0 = 0 \implies \beta = -\alpha
        \]
        Evaluating at, say $x = 1/4$, we find
        \begin{gather*}
          \alpha + \frac{2}{\sqrt{2}}\beta + \frac{2}{\sqrt{2}}\delta = 0 \\
          (1 - \sqrt{2})\alpha = 0
        \end{gather*}
        Only $\alpha = 0$, and therefore $\beta = 0, \delta = 0$
        satisfy this equation. Since only the trivial combination
        yields $\bm{0}$, the functions are linearly independent.
        
        
        
	  \end{enumerate}

	\item Show that the following vectors are linearly independent:
	  \[
	  \bbm a\\0\\0\ebm,\bbm b\\c\\0\ebm,\bbm d\\e\\f\ebm,
	  \]
	  where $a,b,c,d,e,f\in\R$, provided that $a,b,f\neq 0$. Does this set form a basis for $\R^3$? Why or why not?

	  \solution

	  Since $dim(\R^3) = 3$, and we have 3 basis vectors, it suffices
      to show that they are linearly independent. By working
      backwards, we can show that no $\alpha, \beta, \delta \in \R$
      satisfies the equation
      \[
      \alpha \bbm a\\0\\0\ebm + \beta \bbm b\\c\\0\ebm + \delta \bbm d\\e\\f\ebm = \bbm 0\\0\\0 \ebm
      \]
      other than the trivial solution. If we examine only the $x_3$
      component of the sum, we can see that $\delta f = 0 \implies
      \delta = 0$. But in that case, $\beta c = 0$ as well, since the
      second component becomes
      \[
      \beta c + \delta e = \beta c = 0 \implies \beta = 0
      \]
      We can similarly argue that $\alpha$ must 0. Only the trivial
      combination satisfies the relation, the set is linearly
      indepedent, and forms a basis for $\R^3$.

    \end{enumerate}


    \section*{Basis and dimension:}
    \begin{enumerate}[resume]

	\item Find the dimension of and a basis for the following real linear spaces. Justify your answers.
	  \begin{enumerate}
	  \item $S=\left\{\x\in\R^3 : x_1+3x_2-5x_3=0 \right\}$
	  \item $S=\{\x\in\R^3 : x_1=x_3 \}$
	  \item $S = \{p\in\mathcal{P}_3: p(1)=0\}$
	  \item $S = \{p\in\mathcal{P}_3: p(-1)=0, p'(1)=0\}$
	  \end{enumerate}

	  \solution
	  \begin{enumerate}
	  \item By starting with $x_1 = 1$, we can immediately find two
        basis vectors by holding $x_2$ or $x_3$ to 0.
        \[
        \bbm 1 \\ -1/3 \\ 0 \ebm, \bbm 1 \\ 0 \\ -1/5 \ebm
        \]
        We started with $dim(\R^3) = 3$ and added one constraint, so we
        would expect $dim(S) = 2$.

        Checking: TODO!
        
	  \item If we chose $x_1 = 1$ and $x_1 = 0$ as our starting
        points, we can immediately generate two basis vectors
        \[
        \bbm 1\\\alpha\\1 \ebm, \bbm 0\\\beta\\0 \ebm
        \]
        we know that $\beta \neq 0$, if it were, the second basis
        vector would be $\bm{0}$ which is disallowed. We can
        arbitrarily add $[1 0 1]^t$ to the set. To show that we then
        cannot add another basis vector for $\alpha \neq 0$, observe
        that if we did, we could form the sum
        \[
        a \bbm 1\\ 0 \\1 \ebm + b \bbm 1\\ \alpha \\1 \ebm + c \bbm 0\\ 1 \\0 \ebm = 0
        \]
        By selecting $a = -b$ and $c = -\alpha$, we can create a
        non-trivial combination summing to $\bm{0}$. Therefore
        \[
        \bbm 1\\ 0 \\1 \ebm, \bbm 0\\ 1 \\0 \ebm
        \]
        Is a basis for $S$.


      \item $p$ has the form $a + bx + cx^2 + dx^3$. If we evaluate
        this at 1, we find that our constraint is $a + b + c + d =
        0$. By setting $a = 1$, and holding all but one of the other
        components to zero, we can find 3 basis functions
        \[
          1 - x, 1 - x^2, 1 - x^3
        \]
        Checking: TODO!
        
	  \item $p$ has the form $a + bx + cx^2 + dx^3$, which implies
        that $p' = b + 2cx + 3dx^2$. By evaluating our constraints, we
        find that
        \begin{gather*}
          a - b + c - d = 0 \\
          b + 2c + 3d = 0
        \end{gather*}
        Beginning with $a = 1, c = -1$, we can see that by
        substitution, $b + d = 0$ and $b + 3d = 2$. By subtracting, we
        see that $2d = 2 \implies d = 1$, which means that $b = -1$,
        giving us
        \[
        1 - x - x^2 + x^3
        \]
        as our first basis function.

        Choosing $a = 1, b = 1$, we conclude that $c = d$ and $b = -5c
        \implies c = -1/5$ which yields
        \[
        1 + x - \frac{1}{5}x^2 - \frac{1}{5}x^3
        \]
        as our second basis function.

        Checking: TODO!
        
	  \end{enumerate}
    \end{enumerate}

\end{document}
