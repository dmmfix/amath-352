\documentclass[]{article}
\usepackage{amsmath}		% For generic math symbols
\usepackage{amssymb}		% For mathbb
\usepackage{enumerate}		% For lists indexed by letters
\usepackage{bm}				% For bold symbols
\usepackage{enumitem}		% So we can resume counting problem numbers after
							% interrupting with text

\setlength{\parindent}{0pt}	% Turns off indentation


% Set some useful commands
\newcommand{\half}{\frac{1}{2}}
\newcommand{\R}{\mathbb{R}}
\newcommand{\bbm}{\begin{bmatrix}}
\newcommand{\ebm}{\end{bmatrix}}

% Place this command after each problem, before solution (examples below)
\newcommand{\solution}{\vskip 0.5cm \textbf{\large Solution:} \\}


\title{AMATH 352: Problem set 1 solutions}
\author{Your name here}

\begin{document}

\maketitle

\section*{Column Vectors:}

% The enumerate environment is used for numbered lists. Each time you want to
% add another number to the list, use the \item command
\begin{enumerate}

	% First problem
	\item Show that for any $\alpha\in\R$ and any $\bm{x},\bm{y}\in\mathbb{R}^m$, $\alpha(\bm{x}+\bm{y}) = \alpha\bm{x} + \alpha\bm{y}$.


	\solution
	Your solution goes here.

	% Next problem
	\item Show that for every vector $\bm{x}\in\mathbb{R}^m,~ 1\bm{x}=\bm{x}$.

	\solution
	Your solution goes here.
\end{enumerate}


\section*{Norms and inner products:}

\begin{enumerate}[resume]
	
	% Next problem
	\item Determine whether each of the following functions is a norm. Justify your answer, i.e. if you claim it is a norm, show that it satisfies the five criteria discussed in class, and if not, give a concrete example that shows it doesn't satisfy one of the criteria.
	
	% Creating an enumerate environment inside another enumerate environment creates a sub-list
	\begin{enumerate}
		\item $\bm{x}\in\R^3$, $\|\bm{x}\| := \|\bm{x}\|_2-\|\bm{x}\|_1$
		\item $\bm{x}\in\R^3$, $\|\bm{x}\| := \|\bm{x}\|_2 + \|\bm{x}\|_1$
		\item $\bm{x}\in\R^3$, $\|\bm{x}\| := $ the number of nonzero entries in $\bm{x}$.
		\item $\bm{x}\in\R^3$, $\|\bm{x}\| := 4|x_1| + |x_1-x_2+x_3| + |x_2+x_3|$
	\end{enumerate}

	\solution
	\begin{enumerate}
		\item Solution to first part goes here.
		\item Solution to second part goes here.
		\item Solution to third part goes here.
		\item Solution to fourth part goes here.
	\end{enumerate}

	% Next problem
	\item Find and sketch the closed unit ball in $\R^2$ for the infinity norm. Justify your drawing (your answer for this problem should be more than just a drawing).

	\solution
	Your solution goes here.

\end{enumerate}

\end{document}
