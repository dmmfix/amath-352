\documentclass[]{article}
\usepackage{amsmath}		% For generic math symbols
\usepackage{amssymb}		% For mathbb
\usepackage{enumerate}		% For lists indexed by letters
\usepackage{bm}				% For bold letters
\usepackage{enumitem}		% So we can resume counting problem numbers after
							% interrupting with text
\usepackage{hyperref}		% For clickable URL links

\setlength{\parindent}{0pt}	% Turns off indentation


% Set some useful commands
\newcommand{\half}{\frac{1}{2}}
\newcommand{\R}{\mathbb{R}}
\newcommand{\C}{\mathbb{C}}
\newcommand{\bbm}{\begin{bmatrix}}
\newcommand{\ebm}{\end{bmatrix}}
\newcommand{\x}{\bm{x}}
\newcommand{\y}{\bm{y}}
\newcommand{\vspan}{\mathrm{span}}

% Place this command after each problem, before solution (examples below)
\newcommand{\solution}{\vskip 0.5cm \textbf{\large Solution:} \\}

\title{AMATH 352: Problem Set 2}
\author{Your name here}

\begin{document}

\maketitle
{\Large \textbf{Due: Friday January 20, 2017}}
\\

\section*{Norms:}
\begin{enumerate}
	\item Find and sketch the closed unit ball in $\R^2$ for the infinity norm. Justify your drawing (your answer for this problem should be more than just a drawing).

	  \solution
      
	Your solution here
\end{enumerate}

\section*{Real Linear Spaces:}

\begin{enumerate}[resume]
	\item Verify that $\mathbb{C}^2$ (the set of column vectors with two entries which are both in $\mathbb{C}$) is a real linear space, i.e. show that it satisfies the 10 defining properties of a real linear space.

	  \solution
      \begin{enumerate}
      \item $\forall a,b \in \C^2$
        \[
        a + b = \bbm a_1 \\ a_2 \ebm + \bbm b_1 \\ b_2 \ebm = \bbm a_1 + b_1 \\ a_2 + b_2 \ebm
        \]
        Because the complex numbers are closed under addition, each
        $a_i + b_i$ term is also a complex number. The resulting
        vector has 2 elements, and is in $\C^2$, so $\C^2$ is closed
        under addition.
      \item Addition in $\C^2$ is commutative because the complex
          numbers are commutative under addition:
          \[
        a + b = = \bbm a_1 + b_1 \\ a_2 + b_2 \ebm = \bbm b_1 + a_1 \\ b_2 + a_2 \ebm = b + a
        \]
      \item Addition in $\C^2$ is associative because the complex
        numbers are associative:
        \[
        a + (b + c) = \bbm a_1 + (b_1 + c_1) \\ a_2 + (b_2 + c_2) \ebm = \bbm (a_1 + b_1) + c_1 \\ (a_2 + b_2) + c_2 \ebm = (a + b) + c
        \]
      \item The zero vector in $\C^2$ is $\bbm 0 \\ 0 \ebm$:
        \[
        a + \bm{0} = \bbm a_1 + 0 \\ a_2 + 0 \ebm  = \bbm a_1 \\ a_2 \ebm = 0
        \]
      \item Because every element $x \in \C$ has an additive inverse
        $-x \in \C$, every element of $\C^2$ also has an additive
        inverse:
        \[
        a + -a = \bbm a_1 + -a_1 \\ a_2 + -a_2 \ebm = \bbm 0 \\ 0 \ebm = \bm{0}
        \]
      \item Because $\C$ is closed under scalar multiplication,
        $\forall x \in R, z \in C, xz \in \C$, $\C^2$ is similarly
        closed under scalar multiplication
        \begin{gather}
          x \in \R \\
          a \in \C^2 \\
          x a = \bbm x a_1 \\ x a_2 \ebm
        \end{gather}
        The $x a_i$ terms are complex numbers, so this 2-element vector is in $\C^2$.
        
      \item TODO (a + b)x = ax + bx
      \item TODO a (u + v) = au + av
      \item TODO 1 * v = v
      \end{enumerate}
         
	\item Which of the following sets $W$ are subspaces of $V$ (and hence are linear spaces themselves)? Justify your answers with arguments showing they are closed under addition and scalar multiplication or counterexamples showing they are not.
	\begin{enumerate}
		\item $V = \R^4$ and $W = \left\{ \x=\bbm x_1\\x_2\\x_3\\x_4 \ebm\in\R^4: x_1+2x_2 = 0 ~\mathrm{and}~ x_3-x_4=0 \right\}$
		\item $V=C^0(\R,\R)$ and $W = \{ f\in V : f(3) = 2\}$
		\item $V = \mathcal{F}(\R,\R)$ and $W$ is the set of all periodic functions of period 1, i.e. the set of all functions $f$ such that $f(x+1)=f(x)$ for all $x\in\R$.
	\end{enumerate}

	\solution

	\begin{enumerate}
		\item Your solution here
		\item Your solution here
		\item Your solution here
	\end{enumerate}

	\item Explain why the following set is not a real linear space:
	\[
	W=\left\{\bbm x_1\\x_2\\x_3 \ebm\in\R^3 : x_1+x_2x_3=1\right\}.
	\]
	
	\solution
    This set is not closed under addition. If we take two elements of the set
    \[
    a = \bbm 7 \\ -2 \\ 3 \ebm, b = \bbm 1  \\ 0 \\ 0 \ebm
    \]
    we can see that their sum
    \[
    c = a + b = \bbm 7 \\ -2 \\ 3 \ebm + \bbm 1  \\ 0 \\ 0 \ebm = \bbm 8 \\ -2 \\ 3 \ebm
    \]
    is not in $W$. ($8 + -2*3 = 2 \neq 1$)
\end{enumerate}

\section*{Span:}
\begin{enumerate}[resume]
	\item Answer the following (justify your answers):
	\begin{enumerate}
	\item Is $\bbm 1\\7\ebm$ in $\vspan \left(\bbm 1\\2 \ebm,\bbm 3\\1 \ebm \right)$?
		\item Is $\bbm 7\\8\\9\ebm$ in $\vspan\left(\bbm 1\\2\\3 \ebm,\bbm 4\\5\\6 \ebm \right)$?
		\item Is $f(x)=1$ (the constant function that is 1 for any input $x$) in $\vspan(2x^2-2, x+3)$?
	\end{enumerate}
    
	\solution
	\begin{enumerate}
	\item
      $\bbm 1\\7\ebm = 4 \bbm 1\\2 \ebm - \bbm 3\\1 \ebm$, so it is in
      the span. More generally, since the basis vectors are linearly
      independent (observe that $1/3 \neq 2/1$), they span all of $\R^2$.
      
	\item $\bbm 7\\8\\9\ebm = -\bbm 1\\2\\3 \ebm + 2 \bbm 4\\5\\6
      \ebm$, it is in the span of this basis.

	\item $f(x) = 1$ is not in $\vspan(2x^2-2, x+3)$. We can see that
      any linear combination of the basis functions
      \[
      \alpha(2x^2 - 2) + \beta(x + 3)
      \]
      will have non-zero coefficients for $x^2$ and $x$ unless $\alpha
      = 0$ and $\beta = 0$. This would force the constant coefficient
      to 0 as well, yielding $f(x) = 0$ rather than $f(x) = 1$.
	\end{enumerate}
\end{enumerate}


\end{document}
