\documentclass[]{article}
\usepackage{amsmath}		% For generic math symbols
\usepackage{amssymb}		% For mathbb
\usepackage{enumerate}		% For lists indexed by letters
\usepackage{bm}				% For bold letters
\usepackage{enumitem}		% So we can resume counting problem numbers after
% interrupting with text
\usepackage{hyperref}		% For clickable URL links
\usepackage{url}			% So file names won't create hboxes

\setlength{\parindent}{0pt}	% Turns off indentation


% Set some useful commands
\newcommand{\half}{\frac{1}{2}}			% 1/2
\newcommand{\R}{\mathbb{R}}				% Reals symbol
\newcommand{\bbm}{\begin{bmatrix}}		% Begin bmatrix environment
\newcommand{\ebm}{\end{bmatrix}}		% End bmatrix environment
\newcommand{\x}{\bm{x}}					% Bold (vector) x
\newcommand{\y}{\bm{y}}					% Bold (vector) y
\newcommand{\A}{\bm{A}}					% Bold (matrix) A
\newcommand{\vspan}{\mathrm{span}}		% To use the word span in math mode

% Place this command after each problem, before solution (examples below)
\newcommand{\solution}{\vskip 0.5cm \textbf{\large Solution:} \\}


\title{AMATH 352: Problem Set 4}
\author{Dave Moore, dmmfix@uw.edu}

\begin{document}

\maketitle
    {\Large \textbf{Due: Friday February 10, 2017}} \\

    \vskip 1cm

    \section*{Linear functions}
    \begin{enumerate}
	\item The following linear functions can be written as $\bm{f}(\x)=\A\x$ for some matrix $\A$ (not necessarily square). In each case, determine $\A$.
	  \begin{enumerate}
	  \item $\bm{f}(\x) = \bbm 3x_1-2x_2\\4x_2-x_3 \ebm,\quad\forall\x\in\R^3$.
	  \item $\bm{f}(\x) = \bbm x_3\\x_4\\x_2\\x_1 \ebm,\quad\forall\x\in\R^4$.
	  \item $\bm{f}(\x) = 2x_1-4x_3+5x_4,\quad\forall\x\in\R^4$.
	  \item $\bm{f}(\x) = \bbm x_1\cos(\theta)+x_3\sin(\theta)\\x_2\\-x_1\sin(\theta)+x_3\cos(\theta) \ebm,\quad\forall\x\in\R^3$, for some fixed\\ $\theta\in[0,2\pi)$. This type of matrix is called a rotation matrix.
	  \end{enumerate}

	  \solution
	  \begin{enumerate}
	  \item $\bm{A} = \bbm 3 & -2 & 0 \\ 0 & 4 & -1 \ebm$
	  \item $\bm{A} = \bbm 0 & 0 & 1 & 0 \\ 0 & 0 & 0 & 1 \\ 0 & 1 & 0 & 0 \\ 1 & 0 & 0 & 0 \ebm$
	  \item $\bm{A} = \bbm 2 & 0 & -4 & 5 \ebm$
	  \item $\bm{A} = \bbm cos(\theta) & 0 & sin(\theta) \\ 0 & 1 & 0 \\ -sin(\theta) & 1 & cos(\theta) \ebm$
	  \end{enumerate}
    \end{enumerate}


    \section*{Range, Rank, and Nullspace of a matrix:}
    \begin{enumerate}[resume]
	\item Compute the rank, dimension of the nullspace, and a basis of the nullspace for the following matrices:
	  \begin{enumerate}
	  \item $\A = \bbm 1&0&0\\0&1&0\\0&0&1 \ebm$
	  \item $\A = \bbm 0&1&0\\0&0&1\\0&0&0 \ebm$
	  \item $\A = \bbm 1&0\\-1&-1\\0&1 \ebm$
	  \item $\A = \bbm 1&4&7\\2&5&8\\3&6&9 \ebm$
	  \end{enumerate}

	  \solution
	  \begin{enumerate}
	  \item rank($\A$) = 3, since the columns are simply $\bm{e}_{1-3}$,
        which are linearly independent. There is therefore no
        non-trivial null-space for A.
        
	  \item Column 1 is simply the zero vector, which is always
        linearly dependent. Cols 2 and 3 are $\bm{e}_1$ and
        $\bm{e}_2$, which are independent, so rank($\A$) = 2.

        Since there are no non-zero entries in the first column, we
        can see that any vector of the form $ \bbm \alpha & 0 & 0 \ebm^t$ will map
        to $\bm{0}$. Since dim(null($\A$)) = 1 by the rank-nullity
        theorem, $\bbm 1 & 0 & 0 \ebm^t$ is a basis for the nullspace of $\A$.

      \item The columns are not scalar multiples of each other, so
        rank($\A$) = 2. By the rank-nullity theorem, the dimension
        of the nullspace must be zero, since $\A \in \R^{3x2}$.

	  \item rank($\A$) = 2, since
        \[
         -\bbm 1 \\ 2 \\ 3 \ebm + 2 \bbm 4 \\ 5 \\ 6 \ebm - \bbm 7 \\ 8 \\ 9 \ebm = \bm{0}
        \]
        so the columns are not linearly independent. However, $\bbm 1
        & 2 & 3 \ebm^t$ and $\bbm 4 & 5 & 6 \ebm^t$ are not scalar
        multiples, so {\em are} linearly independent. Rank-nullity
        shows that the dimension of the nullspace must be 1, since
        $\A \in \R^{3x3}$.

        We note from the above equation that any vector of the form
        \[
        \bm{x} = \alpha \bbm -1 \\ 2 \\ -1 \ebm
        \]
        will yield $\bm{Ax} = \bm{0}$, so $\bbm -1 & 2 & -1 \ebm^t \in
        null(\A)$. Since we know the dimension of the null space
        is 1, this forms a basis for the null space of $\A$.
        
	  \end{enumerate}

	\item Consider the matrix
	  \[
	  \A = \bm{uv}
	  \]
	  where $\bm{u}\in\R^{n\times 1}$, $\bm{v}\in\R^{1\times m}$, and
      neither $\bm{u}$ nor $\bm{v}$ is the zero vector. If we view
      $\bm{u}$ and $\bm{v}$ as column and row vectors, respectively,
      then $\A$ is called the \textit{outer product} of $\bm{u}$ and
      $\bm{v}$. Find the rank of $\A$ and a basis for the range of
      $\A$.

	  \solution The columns of $\A$ are simply $u_i * \bm{v}$, so they
      are linearly dependent, since they are simply scalar multiples
      of $\bm{v}$. This implies that rank($\A$) = 1, since $\bm{v}
      \neq \bm{0}$, and that $\bm{v}$ is a basis for the range of
      $\A$.



	\item Consider a real matrix $\A\in\R^{m\times n}$ with $m\geq n$. Suppose $\A$ is full rank, i.e. rank$(\A)=n$. Find the nullspace of $\A$.

	  \solution By the rank-nullity theorem, $dim(null(\A)) + rank(\A)
      = n$, since $\A \in \R^{mxn}$. This means that there is no
      non-trivial nullspace for $\A$, that is, $null(\A) =
      \bm{0}$.


    \end{enumerate}

    \section*{Transpose and adjoint of a matrix:}
    \begin{enumerate}[resume]
	\item What can you say about the diagonal entries of a skew-symmetric matrix? How about the diagonal entries of a complex hermitian matrix?

	  \solution For a skew-symmetric matrix, $x_{ij} = -x_{ji}$, so
      any diagonal entry (where $i = j$), must be 0, since only
      $x_{ii} = 0$ satisfies $x_{ii} = -x_{ii}$.

      A complex hermitian matrix satisfies $x_{ij} = {x_{ji}}^*$,
      where $\cdot ^ *$ denotes the complex conjugate. so the diagonal
      entries satisfy $x_{ii} = {x_{ii}}^*$, that is, each diagonal
      entry is equal to it's conjugate. This describes any complex
      number with only a real component, $a + i0, \forall a \in \R$.

    \end{enumerate}


\end{document}
