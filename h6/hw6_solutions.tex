\documentclass[]{article}
\usepackage{amsmath}		% For generic math symbols
\usepackage{amssymb}		% For mathbb
\usepackage{enumerate}		% For lists indexed by letters
\usepackage{bm}				% For bold letters
\usepackage{enumitem}		% So we can resume counting problem numbers after
							% interrupting with text
\usepackage{hyperref}		% For clickable URL links
\usepackage{url}			% So file names won't create hboxes
\usepackage[margin=1in]{geometry}	% Make margins wider


\setlength{\parindent}{0pt}	% Turns off indentation


% Set some useful commands
\newcommand{\half}{\frac{1}{2}}			% 1/2
\newcommand{\R}{\mathbb{R}}				% Reals symbol
\newcommand{\bbm}{\begin{bmatrix}}		% Begin bmatrix environment
\newcommand{\ebm}{\end{bmatrix}}		% End bmatrix environment
\newcommand{\x}{\bm{x}}					% Bold (vector) x
\newcommand{\y}{\bm{y}}					% Bold (vector) y
\newcommand{\A}{\bm{A}}					% Bold (matrix) A
\newcommand{\vspan}{\mathrm{span}}		% To use the word span in math mode
\newcommand{\la}{\langle}				% Left angled bracket <
\newcommand{\ra}{\rangle}				% Right angled bracket >

% Place this command after each problem, before solution (examples below)
\newcommand{\solution}{\vskip 0.5cm \textbf{\large Solution:} \\}


\title{AMATH 352: Problem Set 6}
\author{Your name}

\begin{document}

\maketitle
{\Large \textbf{Due: Friday February 24, 2017}} \\

\section*{Norms and Inner products:}
\begin{enumerate}[resume]
	\item Let $\bm{W}$ be an invertible matrix. Show that the map
	\[
		\|\x\|_{\bm{W}}=\|\bm{Wx}\|_2
	\]
	is a norm on $\R^m$. Is it still a norm if $\bm{W}$ is singular? Why or why not?

	\solution Your solution.


	\item Consider a real square matrix $\bm{M}\in\R^{n\times n}$. Suppose $\bm{M}$ is symmetric and full rank. Furthermore, suppose $\bm{M}$ is positive definite, i.e. it satisfies
	\[
		\forall\x\in\R^n,~\x\neq\bm{0}\implies \x^T\bm{Mx}>0.
	\]
	Show that the map $\la\cdot,\cdot\ra_{\bm{M}}:\R^n\to\R$ given by
	\[
		\la \x,\y\ra_{\bm{M}} =\x^T\bm{My}
	\]
	is an inner product on $\R^n$.

	\solution Your solution.

	\item Show that the function $\la\cdot,\cdot\ra:C^0([-1,1],\R)\to\R$ given by 
	\[
		\la f,g\ra = \int^1_{-1}f(x)g(x)dx
	\]
	is an inner product on $C^0([-1,1],\R)$. Note that $C^0([-1,1],\R)$ denotes the space of continuous functions that take input from [-1,1] and produce output in $\R$.

	\solution Your solution.

\end{enumerate}

\section*{Conditioning:}
\begin{enumerate}[resume]
	\item In this problem we show that orthogonal matrices are ``perfectly conditioned'' in the sense that their condition numbers with respect to the 2-norm are always 1. Suppose $\bm{O}\in\R^{n\times n}$ is an orthogonal matrix.
	\begin{enumerate}
		\item Show that $\|\bm{O}\|_2=1$. (Hint: recall that orthogonal matrices preserve the 2-norms of vectors).
		\item Show that $\bm{O}^T$ is an orthogonal matrix.
		\item Show that $\kappa_2(\bm{O})=1$.
	\end{enumerate}

	\solution
	\begin{enumerate}
		\item Your solution.
		\item Your solution.
		\item Your solution.
	\end{enumerate}
\end{enumerate}

\section*{Operation count:}
\begin{enumerate}[resume]
	\item Find the number of necessary floating point operations required to compute the following operations (using the big-oh notation introduced in class). Explain your reasoning in each case.
	\begin{enumerate}
		\item Compute the sum $\A+\bm{B}$ for $\A,\bm{B}\in\R^{m\times n}$.
		\item Compute the outer product $\bm{uv}^T$ for $\bm{u},\bm{v}\in\R^n$.
		\item Compute the product $\A\x$ for $\x\in\R^n$ and $\A\in\R^{n\times n}$ where $\A$ is upper triangular.
	\end{enumerate}

	\solution
	\begin{enumerate}
		\item Your solution.
		\item Your solution.
		\item Your solution.
	\end{enumerate}


	\item \textbf{Comparing growth rates:} This exercise is meant to give you an idea of how quickly the number of flops required to solve a problem can increase when one increases the problem size, depending on the complexity of the algorithm used. Construct a table comparing $n,n\log_2(n),n^2,n^3,2^n,$ and $n!$ for $n=2,4,8,16,64,512$. (You may need to use something like Wolfram Alpha to compute some of these quantities).

	\solution Your solution.

\end{enumerate}

\end{document}
