\documentclass[]{article}
\usepackage{amsmath}		% For generic math symbols
\usepackage{amssymb}		% For mathbb
\usepackage{enumerate}		% For lists indexed by letters
\usepackage{bm}				% For bold letters
\usepackage{enumitem}		% So we can resume counting problem numbers after
% interrupting with text
\usepackage{hyperref}		% For clickable URL links
\usepackage{url}			% So file names won't create hboxes
\usepackage[margin=1in]{geometry}	% Make margins wider
\usepackage{graphicx}		% For bridge image



\setlength{\parindent}{0pt}	% Turns off indentation


% Set some useful commands
\newcommand{\A}{\bm{A}}					% Bold (matrix) A
\newcommand{\ML}{\bm{L}}				% Bold (matrix) L
\newcommand{\MU}{\bm{U}}				% Bold (matrix) U
\newcommand{\R}{\mathbb{R}}				% Reals symbol
\newcommand{\bbm}{\begin{bmatrix}}		% Begin bmatrix environment
\newcommand{\ebm}{\end{bmatrix}}		% End bmatrix environment
\newcommand{\half}{\frac{1}{2}}			% 1/2
\newcommand{\la}{\langle}				% Left angled bracket <
\newcommand{\ra}{\rangle}				% Right angled bracket >
\newcommand{\vdim}{\mathrm{dim}}		% To use the word span in math mode
\newcommand{\vmax}{\mathrm{max}}		% To use the word span in math mode
\newcommand{\vnull}{\mathrm{null}}		% To use the word span in math mode
\newcommand{\vrange}{\mathrm{range}}		% To use the word span in math mode
\newcommand{\vrank}{\mathrm{rank}}		% To use the word span in math mode
\newcommand{\vspan}{\mathrm{span}}		% To use the word span in math mode
\newcommand{\x}{\bm{x}}					% Bold (vector) x
\newcommand{\y}{\bm{y}}					% Bold (vector) y
\newcommand{\z}{\bm{z}}					% Bold (vector) z

% Place this command after each problem, before solution (examples below)
\newcommand{\solution}{\vskip 0.5cm \textbf{\large Solution:} \\}


\title{AMATH 352: Problem Set 7}
\author{Your name}

\begin{document}

\maketitle
    {\Large \textbf{Due: Friday March 3, 2017}} \\

    \section*{LU factorization:}
    \begin{enumerate}
	\item In deriving the LU factorization we implicitly relied on the fact that the product of two lower triangular matrices is also lower triangular. Prove this result for the specific case of lower triangular matrices in $\R^{3\times3}$. That is prove that the product of two lower triangular $3\times3$ matrices is also lower triangular.
	\item Using the results from lecture, what is the inverse of each of the following matrices?
	  \begin{enumerate}
	  \item $\bbm 1&0&0\\0&1&0\\0&m_1&1 \ebm$
	  \item $\bbm 1&0&0\\m_1&1&0\\m_2&0&1 \ebm$
	  \item $\bbm 1&0&0&0\\0&1&0&0\\0&0&1&0\\0&0&5&1 \ebm \bbm 1&0&0&0\\0&1&0&0\\0&3&1&0\\0&9&0&1 \ebm \bbm 1&0&0&0\\6&1&0&0\\-2&0&1&0\\8&0&0&1\ebm $
	  \end{enumerate}

	  \solution
	  \begin{enumerate}
	  \item $\bbm 1&0&0\\0&1&0\\0&m_1&1 \ebm^{-1} = \bbm 1&0&0\\0&1&0\\0&-m_1&1 \ebm$
	  \item $\bbm 1&0&0\\m_1&1&0\\m_2&0&1 \ebm^{-1} = \bbm 1&0&0\\-m_1&1&0\\-m_2&0&1 \ebm$
	  \item
        $\left(\bbm 1&0&0&0\\0&1&0&0\\0&0&1&0\\0&0&5&1 \ebm \bbm 1&0&0&0\\0&1&0&0\\0&3&1&0\\0&9&0&1 \ebm \bbm 1&0&0&0\\6&1&0&0\\-2&0&1&0\\8&0&0&1\ebm \right)^{-1} =
        \bbm 1&0&0&0\\-6&1&0&0\\2&-3&1&0\\-8&-9&-5&1 \ebm$
	  \end{enumerate}


	\item Compute, by hand, the LU decomposition of the following matrices. Show your work.
	  \begin{enumerate}
	  \item $\bbm 2&1&3\\4&1&2\\0&7&8 \ebm $
	  \item $\bbm 2&1&2&3\\4&0&5&11\\12&10&15&-13\\8&6&12&-1 \ebm $
	  \end{enumerate}

	  \solution
	  \begin{enumerate}
	  \item
        \[\begin{split}
        \bbm 1&0&0 \\ -2&1&0 \\ 0&0&1 \ebm \bbm 2&1&3\\4&1&2\\0&7&8 \ebm &= \bbm 2&1&3\\0&-1&-4\\0&7&8 \ebm \\
        \bbm 1&0&0 \\ 0&1&0 \\ 0&7&1 \ebm \bbm 1&0&0 \\ -2&1&0 \\ 0&0&1 \ebm \bbm 2&1&3\\4&1&2\\0&7&8 \ebm &= \bbm 2&1&3\\0&-1&-4\\0&0&-20 \ebm \\
        \ML = \bbm 1&0&0 \\ 2&1&0 \\ 0&-7&1 \ebm&,~\MU = \bbm 2&1&3\\0&-1&-4\\0&0&-20 \ebm
        \end{split}\]
	  \item
        \[\begin{split}
        \bbm 1&0&0&0 \\ -2&1&0&0 \\ -6&0&1&0 \\ -4&0&0&1 \ebm \bbm 2&1&2&3\\4&0&5&11\\12&10&15&-13\\8&6&12&-1 \ebm &= \bbm 2&1&2&3\\0&-2&1&5\\0&4&3&-31\\0&2&4&-13 \ebm \\
        \bbm 1&0&0&0 \\ 0&1&0&0 \\ 0&2&1&0 \\ 0&1&0&1 \ebm \bbm 1&0&0&0 \\ -2&1&0&0 \\ -6&0&1&0 \\ -4&0&0&1 \ebm \bbm 2&1&2&3\\4&0&5&11\\12&10&15&-13\\8&6&12&-1 \ebm &= \bbm 2&1&2&3\\0&-2&1&5\\0&0&5&-21\\0&0&5&-8 \ebm \\
        \bbm 1&0&0&0 \\ 0&1&0&0 \\ 0&0&1&0 \\ 0&0&-1&1 \ebm \bbm 1&0&0&0 \\ 0&1&0&0 \\ 0&2&1&0 \\ 0&1&0&1 \ebm \bbm 1&0&0&0 \\ -2&1&0&0 \\ -6&0&1&0 \\ -4&0&0&1 \ebm \bbm 2&1&2&3\\4&0&5&11\\12&10&15&-13\\8&6&12&-1 \ebm &= \bbm 2&1&2&3\\0&-2&1&5\\0&0&5&-21\\0&0&0&13 \ebm \\
        \ML = \bbm 1&0&0&0 \\ 2&1&0&0 \\ 6&-2&1&0 \\ 4&-1&1&1 \ebm &,~\MU = \bbm 2&1&2&3\\0&-2&1&5\\0&0&5&-21\\0&0&0&13 \ebm
        \end{split}\]
	  \end{enumerate}

	\item Given the following LU factorization $\A=\bm{LU}$ and the vectors $\x,\y$:
	  \[
	  \bm{L}=\bbm 1&0&0\\1&1&0\\0&1&1\ebm,\quad \bm{U}=\bbm 1&0&1\\0&1&1\\0&0&1\ebm,\quad \x=\bbm 1\\0\\1\ebm, \quad \y=\bbm -1\\1\\1\ebm
	  \]
	  compute
	  \[
	  \A^{-1}\x + \A^{-2}\y
	  \]
	  without forming $\A$, $\A^2$, $\A^{-1}$, or $\A^{-2}$ explicitly. Show your work.

	  \solution
      \[\begin{split}
      \A^{-1}\x &= (\bm{LU})^{-1}\x \\
      &= \bm{U}^{-1}\bm{L}^{-1}\x \\
      
      \end{split}\]
	  Your solution

    \end{enumerate}

\end{document}
